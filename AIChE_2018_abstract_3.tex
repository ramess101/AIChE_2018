\documentclass[11pt,a4paper]{article}
\usepackage{graphicx}
% uncomment according to your operating system:
% ------------------------------------------------
\usepackage[latin1]{inputenc}    %% european characters can be used (Windows, old Linux)
%\usepackage[utf8]{inputenc}     %% european characters can be used (Linux)
%\usepackage[applemac]{inputenc} %% european characters can be used (Mac OS)
% ------------------------------------------------
\usepackage{authblk}
\usepackage[superscript]{cite}
\usepackage[document]{ragged2e}
\usepackage[T1]{fontenc}   %% get hyphenation and accented letters right
\usepackage{mathptmx}      %% use fitting times fonts also in formulas
% do not change these lines:
\pagestyle{empty}                %% no page numbers!
\usepackage[left=35mm, right=35mm, top=15mm, bottom=20mm, noheadfoot]{geometry}
%% please don't change geometry settings!


% begin the document
\begin{document}
	\thispagestyle{empty}
	%make title bold and 14 pt font (Latex default is non-bold, 16 pt)
	\title{\Large \textbf{Beyond united-atom Lennard-Jones: Reliable prediction of high pressure viscosities}}
			%The Industrial Fluid Properties Simulation Challenge:
			% An arduous test of force field transferability}}
			% Accuracy of force field parameterized with high pressure thermodynamic properties}}
			% Transferability of non-bonded potential parameterized with isooctane REFPROP equation-of-state}}
			%Force field parameterized with isooctane REFPROP equation of state
			%Accuracy of force field parameterized with high pressure thermodynamic properties
	\author[1]{\large {Andrei Kazakov}}
	\author[2]{\large {J. Richard Elliott}}
	\author[2]{\large {S. Mostafa Razavi}}
	\author[1]{\large {\underline{Richard Messerly}}}%%[12 pt regular, presenting speaker underlined]

	
	\affil[1]{\textit{Thermodynamics Research Center (TRC), National Institute of Standards and Technology (NIST),
			Boulder, Colorado, 80305, USA}}
	%\affil[2]{\textit{Department of Chemical and Biological Engineering, University of Colorado, Boulder, Colorado, 80309, USA}}
	\affil[2]{\textit{Department of Chemical and Biomolecular Engineering, The University of Akron, Akron, Ohio, 44325, USA}}
	
	\date{} % <--- leave date empty
	\maketitle\thispagestyle{empty} %% <-- you need this for the first page
	\begin{center}
		\title{\textbf{ABSTRACT}}\centering{}
	\end{center}
	\justify
	
	%AIChE description:
	
	%	The Industrial Fluid Properties Simulation Challenge
	%Contributions describing entries in the Industrial Fluid Properties Simulation Challenge 
	
	
%	1. Accurate prediction of viscosity necessitate reliable non-bonded potentials that extrapolate well to high pressures, i.e. that they predict accurate densities and viscosities
%	2. Isooctane is used as a surrogate compound
%	3. CH3, CH2, CH, and C parameters are optimized simultaneously to REFPROP
	
	Accurate prediction of viscosity $(\eta)$ at high pressures $(P)$ necessitates an extremely reliable force field for at least two reasons. First, the viscosity at a given density $(\rho)$ is highly sensitive to the potential function form and associated parameters, especially those of the non-bonded interactions. Second, the viscosity depends strongly on the predicted density, which is also very sensitive to the force field. 
	
	As preliminary work suggests that traditional united-atom Lennard-Jones $n$-6 force fields are not capable of accurately predicting high pressure viscosities, a novel anisotropic-united-atom model is proposed. To develop a highly accurate force field, the CH$_3$, CH$_2$, CH, and C non-bonded parameters are optimized simultaneously using a large data set, consisting of $P\rho T$ and caloric properties over a wide range of state points, with particular emphasis at high pressures, for several normal and branched alkanes. Since the challenge compound is 2,2,4-trimethylhexane (TMH), we use 2,2,4-trimethylpentane (TMP, a.k.a. isooctane) as a surrogate molecule to refine the non-bonded parameters and, consequentially, to improve transferability. 
	
	%Specifically,are obtained empirically by minimizing the deviation of $P\rho T$ and caloric properties for TMP over a wide range of state points, with particular emphasis at high pressures. 
	
	% an advanced configurational reweighting scheme. 
	
	% implement a configurational reweighting scheme to dramatically reduce the computational cost to parameterize the non-bonded potential.
	
	% parameterization. A configurational reweighting scheme  The CH$_3$, CH$_2$, CH, and C non-bonded parameters are optimized simultaneously by implementing a configurational reweighting scheme that dramatically reduces computational costs for force field parameterization. 
	
	%This high-dimensional parameterization is possible by reweighting configurations using Multistate Bennett Acceptance Ratio (MBAR) and by including several properties over a wide range of state points.  
	
	An essential aspect of the challenge is to provide meaningful estimates of uncertainty. For this reason, uncertainties in the predicted TMH viscosity are quantified using three different methods. First, we estimate systematic bias in the force field by comparing the simulated and REFPROP $\eta$ values for the surrogate compound, TMP, at the challenge temperature and pressures. Second, we account for the uncertainty in $\eta$ that is associated with uncertainties in $\rho$ for a given $P$. Third, we implement Bayesian inference to quantify and propagate the uncertainty in the force field non-bonded parameters. 
	
\end{document}
